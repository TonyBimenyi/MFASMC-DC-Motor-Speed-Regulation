\documentclass[journal,onecolumn]{IEEEtran}
\renewcommand{\baselinestretch}{1.45}
\usepackage{mathrsfs}
\usepackage{booktabs}
\usepackage{cite}

\usepackage{amssymb}
\usepackage{amsmath}
\usepackage{graphics}
\usepackage{color}
\usepackage{epsfig}
\usepackage{graphicx}
\usepackage{amsthm}
\usepackage{subfigure}
\allowdisplaybreaks[4]

\newtheorem{Lem}{Lemma}
\newtheorem{Thm}{Theorem}
\newtheorem{defn}{Definition}
\newtheorem{corollary}{Corollary}
\newtheorem{rem}{Remark}
\newtheorem{assumption}{Assumption}
\newtheorem{procedure}{Procedure}
\newtheorem{algorithm}{Algorithm}
\makeatletter
\newcommand{\rmnum}[1]{\romannumeral #1}
\newcommand{\Rmnum}[1]{\expandafter\@slowromancap\romannumeral #1@}
\makeatother

\title{\LARGE Model-Free Adaptive Iterative Learning Containment Control for Nonlinear Multiagent Systems Under Sparse Sensor Attacks}

\author{pzq}

\begin{document}
 \maketitle

\begin{abstract}
In this work, the model-free adaptive iterative learning control approach is used to study the containment control problem of nonlinear multiagent systems under sparse  sensor attacks. Firstly, the nonlinear dynamics of each agent is transformed into an approximate linear model along the iterative axis by using the compact form dynamic linearization (CFDL) method. Secondly, the detection mechanism and adaptive switching strategy are designed for the multi-source mixed attacks on sensors, which greatly alleviates the negative impact of sparse sensor attacks on the system. Subsequently, the distributed model-free adaptive iterative learning control algorithm is presented to ensure that the output of each follower is constrained within the convex hull formed by the output of leaders. Then, the convergence of the proposed control algorithm is analyzed and extended to the switching topology. Finally, the control effect is verified by several simulation examples.
\end{abstract}

\begin{keywords} Containment control, sparse sensor attacks, model-free adaptive iterative learning control, nonlinear multiagent systems.
\end{keywords}



\section{Introduction}\label{section:1}
During the last few decades, with the increase of task complexity and data volume, the intelligent individual has been gradually difficult to complete the needs of the system, so the multiagent systems came into being. Due to excellent collaboration ability and distributed decision-making mechanism, multiagent systems have been widely used in autonomous vehicles \cite{1}, robot collaboration \cite{2}, Internet of Things \cite{3}-\cite{4} and distributed sensor networks \cite{5}. Containment control is considered to be an important problem in multiagent systems cooperation. The objective of containment control is to find a suitable strategy so that all followers can converge into a convex hull composed of multiple leaders. So far, a great deal of research has been done on containment problems with different system models \cite{6}-\cite{9}.

The above work is carried out on the basis of a common premise. In other words, the system model is available. However, with the progress of science and technology, the accurate system model becomes difficult to obtain, which makes the research of the problem into a bottleneck. At this time, some scholars introduced model-free adaptive control methods. The method only relies on the real-time measurement data of the controlled system, and does not rely on any mathematical model information of the controlled system. In \cite{10}-\cite{12}. In the above literature, model-free adaptive control shows good control effect and robustness. Unfortunately, model-free adaptive control only works in the time domain.

However, in engineering practice, many systems perform the same control tasks repeatedly in a finite time interval. The time domain control method does not have the ability to learn from the past repeated operations, that is, the control error is also repeated. In contrast, the iterative learning control method can reduce the system error by using the memory device to store the past duplicate data to correct the current control input. Therefore, combining the characteristics of model-free adaptive control and iterative learning control, the design and analysis methods of model-free adaptive iterative learning control are proposed.


In the present work, via ISMC technology, the fault-tolerant problem
is studied for a class of linear continuous systems based on
simplified deadzone model with respect to actuator dead-zone failure
and unmatched external perturbations. First of all, the deadzone is
modeled as a combination of a line and a bounded disturbance-like
term. The nonlinear control law is then designed to guarantee that
sliding mode can be achieved. And the bounded disturbance-like term
which enters the state equation at the same point as the control
input can be rejected completely. What's more, a novel bounded real
lemma with allowing extra degree of freedom for design is presented
by using the well-known Projection Lemma. Further, based on this new
lemma, the state feedback controller given by linear matrix
inequality (LMI) is presented such that the resulting closed-loop
system in the sliding mode is asymptotically stable.It is worth
noting that the deadzone failure can be completely compensated and
the system is more robust against unmatched disturbances than using
$H_\infty$ control alone. At last, a numerical example is presented
to show the effectiveness of the proposed scheme.


The notations used throughout this paper are fairly standard.
 $R^n$ denotes the n-dimensional Euclidean space, $R^{m\times n}$ is the
 set of all $m \times n$ real matrices. In addition, we use '*' as
 an ellipsis for the terms that are introduced by symmetry. $I$
 means the unit matrix with appropriate dimensions, $I_n$ is the
 identity matrix of order n.
 $\otimes$ is the Kronecker product of two matrices. The direct sum
 of matrices $A_i, i=1,2,\cdots, n$ will be denoted as
 $A_1\oplus A_2\oplus \cdots \oplus
  A_n \triangleq diag\{A_1,A_2,\cdots,A_n\}$.
In addition, for a matrix $M$, $He\{M\}$ denotes $M+M^T$.%

Secondly, an attack defense scheme is designed for the multi-source mixed attacks occurring in the sensor channel. The scheme is composed of detection mechanism and adaptive switching strategy, which can reduce the negative effects of sparse sensor attacks by switching channel in time after detecting abnormal behavior.

\section{System Description and Problem Statement}

\subsection{Deadzone Model}
Deadzone is a static input-output relationship which for a range of
input values gives no output. Once the output appears, the slope
between the input and the output is constant. The analytical
expression of the deadzone characteristic is

\begin{equation}
\label{model 1}
y_i^l(t+1,k)=f_i^l(y_i^l(t,k),\cdots,y_i^l(t-n_y,k),u_i(t,k),\cdots,u_i(t-n_u,k))
\end{equation}

\begin{equation}
\label{model 2}
\Delta y_i^l(t+1,k)=\phi_i(t,k)\Delta u_i(t,k)
\end{equation}

\begin{equation}
\label{model 3}
e_i^l(t,k)=max\left\{y_{n+1}(t)-y_i^l(t,k),\cdots,y_{n+p}(t)-y_i^l(t,k)\right\}
\end{equation}

\begin{equation}
\label{model 4}
g_i^l(t,k)=\begin{cases}
1,& \text{if }\left|e_i^l(t,k)\right|\ge\hbar  \\
0,& \text{if }\left|e_i^l(t,k)\right|< \hbar
\end{cases}
\end{equation}

\begin{equation}
\label{model 5}
\tilde{y}_i(t,k)=\left[\imath_i^1(t),\imath_i^2(t),\cdots,\imath_i^x(t)\right]\begin{bmatrix}
y_i^1(t,k)\\
y_i^2(t,k)\\
\vdots\\
y_i^x(t,k)
\end{bmatrix}
\end{equation}

\begin{equation}
\label{model 6}
\imath_i^l(t)=\begin{cases}
1,\ \mathcal{Q}_i^1(t)\ne0\ \cap\cdots\cap\mathcal{Q}_i^{l-1}(t)\ne 0\cap\mathcal{Q}_i^{l}(t)=0\\
0,\ others
\end{cases}
\end{equation}

\begin{equation}
\label{model 7}
\mathcal{Q}_i^l(t)=\sum_{t=1}^{T}g_i^l(t,k)
\end{equation}

\begin{equation}
\label{model 8}
\xi_i(t,k)= {\textstyle \sum_{j\in\mathbb{N}_j}}a_{ij}(\tilde{y}_j(t,k)-\tilde{y}_i(t,k))+\sum_{q=n+1}^{n+p} a_{iq}(y_q(t)-\tilde{y}_i(t,k))
\end{equation}

\begin{align}
\label{model 9}
\hat\phi_i(t,k)=&\hat\phi_i(t,k-1)+\frac{\eta\Delta u_i(t,k-1)}{\mu+\left|\Delta u_i(t,k-1)\right|^2}
\notag \\& \times (\Delta \tilde{y}_i(t+1,k-1)-\hat\phi_i(t,k-1)\Delta u_i(t,k-1))
\end{align}

\begin{align}
\label{model 10}
\hat\phi_i(t,k)=\hat\phi_i(t,1),\ if\left|\hat\phi_i(t,k)\right|\le\varsigma \ or\ \left|\Delta u_i(t,k-1)\right|\le \varsigma \ or\
sign(\hat\phi_i(t,k))\ne sign(\hat\phi_i(t,1))
\end{align}

\begin{align}
\label{model 11}
u_i(t,k)=u_i(t,k-1)+\frac{\rho\hat\phi_i(t,k)}{\lambda+\left|\hat\phi_i(t,k)\right|^2}\xi_i(t+1,k-1)
\end{align}


The key features of the deadzone in the control problems
investigated in this paper are
\begin{enumerate}
\item[{\rm  A1}]The deadzone output $\Phi (u(t))$ is not available for
measurement.
\item[{\rm  A2}]The deadzone slopes in positive and negative region
are same, i.e. ${m_r} = {m_l} = m$.
\item[{\rm  A3}]The deadzone parameters $b_r$,$b_l$ and $m$ are unknown,
but their signs are known: ${b_r} > 0,{b_l} < 0,m > 0$.
\item[{\rm  A4}]The deadzone parameters $b_r$,$b_l$ and $m$ are bounded,
i.e. there exist known constants $b_{r\min }$, $b_{r\max }$,
$b_{l\min }$, $b_{l\max }$, $m_{\min }$, $m_{\max }$ such that${b_r}
\in [{b_{r\min }},{b_{r\max }}],{b_l} \in [{b_{l\min }},{b_{l\max
}}],m \in [{m_{\min }},{m_{\max }}]$.
\end{enumerate}


\begin{rem}\label{re1}


Assumption(A1) is common in practical systems, such as servomotors
and servo valves. If $\Phi (u(t))$ is measurable, the control of
deadzone will be relatively easy and will not be discussed in this
paper. Assumption(A2) is generally adopted in the literature (see,
for example, \cite{12,13}) and can commonly be met in the industrial
systems. Assumptions (A3) and (A4) are physically satisfied in real
plants.
\end{rem}

From a practical point of view, we can re-define model
(\ref{Eq:deadzone model 1}) as a simplified deadzone model\cite{4}
which makes the design of controller simpler\cite{xluo09}.
\begin{equation}
\label{Eq:deadzone model 2}
\begin{array}{rcl}
\Phi (u(t)) = mu(t) + d(u(t))
\end{array}
\end{equation}
where $ m $ is called the general slope of the deadzone, $\Phi
(u(t))$ can be calculated from (\ref{Eq:deadzone model 1}) and
(\ref{Eq:deadzone model 2}) as,

\begin{equation}
\label{Eq:deadzone model 3}
\begin{array}{rcl}
\ d(u(t)) = \left\{
\begin{array}{l}
  - m{b_r}{\kern 1pt} {\kern 1pt} {\kern 1pt} {\kern 1pt} {\kern 1pt} {\kern 1pt} {\kern 1pt} {\kern 1pt} {\kern 1pt} {\kern 1pt} {\kern 1pt} {\kern 1pt} {\kern 1pt} {\kern 1pt} {\kern 1pt} {\kern 1pt} {\kern 1pt} {\kern 1pt} {\kern 1pt} {\kern 1pt} {\kern 1pt} {\kern 1pt} {\kern 1pt} {\kern 1pt} {\kern 1pt} {\kern 1pt} {\kern 1pt} {\kern 1pt} {\kern 1pt} {\kern 1pt} {\kern 1pt} for{\kern 1pt} {\kern 1pt} {\kern 1pt} {\kern 1pt} u(t) \ge {b_r} \\
  - mu(t){\kern 1pt} {\kern 1pt} {\kern 1pt} {\kern 1pt} {\kern 1pt} {\kern 1pt} {\kern 1pt} {\kern 1pt} {\kern 1pt} {\kern 1pt} {\kern 1pt} {\kern 1pt} {\kern 1pt} {\kern 1pt} {\kern 1pt} {\kern 1pt} {\kern 1pt} {\kern 1pt} {\kern 1pt} {\kern 1pt} {\kern 1pt} for{\kern 1pt} {\kern 1pt} {\kern 1pt} {\kern 1pt} {b_l} < u(t) < {b_r} \\
  - m{b_l}{\kern 1pt} {\kern 1pt} {\kern 1pt} {\kern 1pt} {\kern 1pt} {\kern 1pt} {\kern 1pt} {\kern 1pt} {\kern 1pt} {\kern 1pt} {\kern 1pt} {\kern 1pt} {\kern 1pt} {\kern 1pt} {\kern 1pt} {\kern 1pt} {\kern 1pt} {\kern 1pt} {\kern 1pt} {\kern 1pt} {\kern 1pt} {\kern 1pt} {\kern 1pt} {\kern 1pt} {\kern 1pt} {\kern 1pt} {\kern 1pt} {\kern 1pt} {\kern 1pt} {\kern 1pt} {\kern 1pt} for{\kern 1pt} {\kern 1pt} {\kern 1pt} u(t) \le {b_l}{\kern 1pt}  \\
 \end{array} \right.\
\end{array}
\end{equation}

From Assumptions (A2) and (A4), one can conclude that $d(u(t))$ is
bounded, and satisfies $\left| {d(u(t))} \right| \le \bar d$, where
$\bar d$  is the upper-bound, which can be chosen as $\bar d$ =
$\max \{ {m_{\max }}{b_{r\max }}$,  $-{m_{\max }}{b_{l\min }}\} $,
where $b_{l\min }$ carries a negative value.







\subsection{System Model}


Consider a linear time-invariant continuous system in the form of
\begin{equation}
\label{Eq:deadzone system}
\begin{array}{rcl}
   \dot{x}(t) &=& Ax(t)+B\Phi (u(t))+Ew(t)


\end{array}
\end{equation}
 where $x(t)\in R^n$ is the state vector,
$w(t)\in R^r$ is an exogenous disturbance in $\mathcal
{L}_2[0,\infty)$, $u(t)\in R^m$ is the control input, and$z(t)\in R
^q$ is the regulated output, respectively. The system matrices $A$,
$B$, $E$, $C$, and $D$ are known constant matrices of appropriate
dimensions.

Then, the resulting  model of continuous-time system in
(\ref{Eq:deadzone system}) considering the deadzone model
(\ref{Eq:deadzone model 2}) and (\ref{Eq:deadzone model 3}) is given
by
\begin{equation}
\label{Eq:equivalent system}
\begin{array}{rcl}
   \dot{x}(t) &=& Ax(t)+Bmu(t) + Bd(u(t))+Ew(t)


\end{array}
\end{equation}

\begin{rem}\label{rem4}
 It is obvious that there is no consideration about the deadzone
failure when $m=1$, $d(u(t))=0$.
\end{rem}

\subsection{Problem Statement}
The problem of interest in this paper is to develop
 a sliding mode controller $u(t)$ such that the deadzone failure can be compensated completely and
 the closed-loop system in the sliding mode(\ref{Eq:closed sliding mode dynamics}) is
asymptotically stable, and is more robust against unmatched
disturbances than using $H_\infty$ control alone.




To facilitate control system design, the following lemmas are
presented and will be used in the later developments.
 \begin{Lem} \label{le2} (Projection Lemma)\cite{JiaYM, GahinetP} Given a symmetric matrix
 $\Phi\in R^{n\times n}$ and two matrices $T_1$, $T_2$ of column
 dimension $n$, there exists a matrix $X$ that satisfies
\begin{equation*}
\begin{array}{l}
\Phi+He\{T_1^TX^TT_2\}<0,
\end{array}
\end{equation*}
 if and only if the following projection inequalities with respect
 to $X$ are satisfied.
\begin{equation*}
\begin{array}{l}
N^T_{T_1}\Phi N_{T_1}<0, N^T_{T_2}\Phi N_{T_2}<0,
\end{array}
\end{equation*}
where $N_{T_1}$ and $N_{T_2}$ denote arbitrary bases of the
null-spaces of $T_1$ and $T_2$, respectively.
\end{Lem}


 \begin{Lem}\label{le3} For a prescribed scalar $\gamma>0$,
the system with the transfer function as
$T(s)=\bar{C}(sI-\bar{A})^{-1}\bar{E}$ is asymptotically stable, and
$\|T(s)\|_{\infty}<\gamma$ if and only if  there exists matrix
$P=P^T>0$
 such that
 \begin{equation}\label{Eq:Lem_33}\left(
   \begin{array}{ccc}
     1 & 1 & 1 \\
     1 & 1 & 1 \\
     1 & 1 & 1 \\
   \end{array}
 \right)
 \end{equation}


\begin{equation}\label{Eq:Lem_3}\left[
\begin{array}{cccc}
PA_{cl}+A_{cl}^TP&PB_{cl}&C_{cl}^T\\
{*}&-\gamma^2I&D_{cl}^T\\
{*}&{*}&-I
\end{array}\right]<0.
\end{equation}
 \end{Lem}



There is no doubt that the standard bounded real lemma is given by a
strict matrix inequality in Lemma \ref{le3}. However, there is a
drawback of the above lemma. The construct products between the
Lyapunov matrix and the system's dynamic matrices will introduce
some conservativeness. Taking into account the above drawback, a
novel bounded real lemma  is obtained by using Projection Lemma.








\section{ Main Result}



Consider a sliding mode controller as follows
\begin{equation}
\label{Eq:Controller}
\begin{array}{rcl}
u(x,t)&=&u_0(x,t)+u_1(x,t)\\[1mm]
\end{array}
\end{equation}
 The nominal control $u_0(x,t)$ is responsible for the performance of
 the nominal system; $u_1(x,t)$ is a discontinuous control action
 that rejects the perturbations by ensuring the sliding
 motion. Usually,the discontinuous control and the linear control are selected as
 \begin{equation}
\label{Eq:nonlinear Controller}
\begin{array}{rcl}
u_1(x,t)&=&-\rho(x,t)(GB)^{-1}{s(x,t)\over \|s(x,t)\|}\\[1mm]
\end{array}
\end{equation}
and
\begin{equation}
\label{Eq:linear Controller}
\begin{array}{rcl}
u_0(x,t)&=&Kx(t)\\[1mm]
\end{array}
\end{equation} respectively.




Now the integral sliding surface is chosen to be
 \begin{equation}
\label{Eq:sliding surface}
\begin{array}{rcl}
s(x,t)&=&G[x(t)-x(t_0)-\int_{t_0}^t(Ax(\tau)+B{m_{min}}u_0(\tau))\,dt]\\[1mm]
\end{array}
\end{equation}
where $s\in R^m$ , $u_0(t) \in R^m$ designed by state feedback
 is a control input to guarantee unmatched disturbance attenuation once the system is in the sliding
 mode, and $G$ is chosen as $B^+$ to avoid amplification of the
 unmatched disturbances \cite{24}.


Differentiating (\ref{Eq:sliding surface}) with respect to time and
substituting $\dot x(t)$ from (\ref{Eq:equivalent system}) and $u(t)
$ from  (\ref{Eq:Controller})obtains
\begin{equation*}
\label{Eq:sliding surface Difference}
\begin{array}{rcl}
\dot{s}(x,t)&=&GB((m-{m_{min}})u_0+mu_l+d(t))+GEw(t)\\[1mm]
\end{array}
\end{equation*}
then
\begin{equation*}
\label{Eq:1}
\begin{array}{rcl}
mu_{leq}=-{m_{min}}u_0-d(t)-{(GB)}^{-1}GEw(t)\\[1mm]
\end{array}
\end{equation*}
we can obtain the sliding mode dynamics

\begin{equation}
\label{Eq:open sliding mode dynamics}
\begin{array}{rcl}
   \dot{x}(t)&=&Ax(t)+(I-B{(GB)}^{-1}G)Ew(t)+B{m_{min}}u_0(t)

\\[1mm]
   z(t)&=& Cx(t)+Du_0(t)
\end{array}
\end{equation}


\begin{rem}\label{re2}

The novelty of the paper lies in that the way to treat the deadzone
failure. The bounded disturbance-like term $d(u(t))$ which enters
the state equation at the same point as the control input can be
rejected completely by ISMC technology, and the linear part of the
deadzone will be compensated by state feedback controller in the
following design.

\end{rem}

 Consider the system
(\ref{Eq:open sliding mode dynamics}) associated with the control
law (\ref{Eq:linear Controller}), the closed-loop system in the
sliding mode can be expressed as follows
\begin{equation}
\label{Eq:closed sliding mode dynamics}
\begin{array}{rcl}
 \dot{x}(t)&=&A_{cl}x(t)+B_{cl}w(t)\\[1mm]
  z(t)&=&C_{cl}x(t)
\end{array}
\end{equation}
where
\begin{equation*}
A_{cl}=A+B{m_{min}}K, B_{cl}=(I-B{(GB)}^{-1}G)E, C_{cl}=C+DK.
\end{equation*}


The transfer function matrix of the closed-loop system in
(\ref{Eq:closed sliding mode dynamics}) in the sliding mode is given
by
\begin{equation}
\label{EQ:Tran form}
\begin{array}{l}
T(s)=C_{cl}(sI-A_{cl})^{-1}B_{cl}.
\end{array}
\end{equation}






Now we will focus on obtaining the state feedback $H_{\infty}$
controller design conditions on the sliding surface and the
achievement conditions.

First, we utilize the following improved bounded real
lemma\cite{JiaYM}, which is crucial to our derivation.
 \begin{Thm}\label{thm1}  For a prescribed scalar $\gamma>0$, the
 sliding dynamics on $ s(x,t)=0 $ with the transfer function as $T(s)=\bar{C}(sI-\bar{A})^{-1}\bar{E}$
  is asymptotically stable, and
satisfies the norm constraint $\|T(s)\|_{\infty}<\gamma$ if and only
if there exist matrices $Q=Q^T>0$, $Z$ and a sufficiently small
scalar $\epsilon>0$ such that
\begin{equation}\begin{array}{l}
\label{Eq:thm 1 condition} \left[\begin{array}{ccccc}
Q-He\{Z\}&Z^T(I+\epsilon A_{cl}) &0&\sqrt{\epsilon}Z^TB_{cl}\\{*}&-Q&\sqrt{\epsilon}C_{cl}^T&0\\
{*}&{*}&-\gamma^2 I&D_{cl}
\\
{*}&{*}&{*}& - I
\end{array}\right]<0.\end{array}
\end{equation}
 \end{Thm}
\begin{proof}
By suitable block-row-column exchanges of (\ref{Eq:thm 1
condition}), it is easy to obtain
\begin{equation}\begin{array}{l}
\label{Eq:Thm 1 p1} \left[\begin{array}{ccccc}-
I&\sqrt{\epsilon}B_{cl}^TZ&0&D_{cl}^T\\{*}&Q-Z-Z^T&Z^T(I+\epsilon
A_{cl})
&0\\{*}&{*}&-Q&\sqrt{\epsilon}C_{cl}^T\\{*}&{*}&{*}&-\gamma^2 I
\end{array}\right]<0.\end{array}
\end{equation}

 Obviously, the matrix inequality (\ref{Eq:Thm 1 p1}) can
be equivalently rewritten as
\begin{equation}\begin{array}{l} \label{Eq:Thm 1 p2}
\left[\begin{array}{ccccc}- I&
0&0&D_{cl}^T\\{*}&Q&0&0\\{*}&{*}&-Q&\sqrt{\epsilon}C_{cl}^T\\{*}&{*}&{*}&-\gamma^2I
\end{array}\right]+He\left\{\left[\begin{array}{c}0\\I\\0\\0\end{array}\right]Z^T\left[\begin{array}{c}
\sqrt{\epsilon}B_{cl}^T\\-I\\I+\epsilon
A_{cl}^T\\0\end{array}\right]^T\right\}<0\end{array}
\end{equation}

Noting that explicit bases of the  null-spaces of
$\left[\begin{array}{ccccc}0&I&0&0\end{array}\right]$ and
$\left[\begin{array}{ccc} \sqrt{\epsilon}B_{cl}&-I&I+\epsilon
A_{cl}\end{array}\right. \\
\left.\begin{array}{cc}&0\end{array}\right]$ are given as
$N_1=\left[\begin{array}{cccc}I&0&0\\0&0&0\\0&I&0\\0&0&I\end{array}\right]$
 and
$N_2=\left[\begin{array}{cccc}I&0&0\\\sqrt{\epsilon}B_{cl}&I+\epsilon
A_{cl}&0\\0&I&0\\0&0&I\end{array}\right]$.

Then, canceling the matrix $Z$ from (\ref{Eq:Thm 1 p2}) by Lemma
\ref{le2} leads to
\begin{equation*}\begin{array}{l}
N_1^T\Psi N_1=\left[\begin{array}{cccc}- I&
0&D_{cl}^T\\{*}&-Q&\sqrt{\epsilon}C_{cl}^T\\{*}&{*}&-\gamma^2 I
\end{array}\right]<0,\end{array}
\end{equation*}
\begin{equation}\begin{array}{l} \label{Eq:Thm 1 p4}N_2^T\Psi
N_2=\left[\begin{array}{cccc}- I+\epsilon
B_{cl}^TQB_{cl}&\sqrt{\epsilon}B_{cl}^TQ\mathcal
{A}^T&D_{cl}^T\\{*}&\mathcal {A}Q\mathcal
{A}^T-Q&\sqrt{\epsilon}C_{cl}^T
\\{*}&{*}&-\gamma^2
I
\end{array}\right]<0,\end{array}
\end{equation}
where $\Psi=\left[\begin{array}{ccccc}-I&
0&0&D_{cl}^T\\{*}&Q&0&0\\{*}&{*}&-Q&\sqrt{\epsilon}C_{cl}^T\\{*}&{*}&{*}&-\gamma^2
I
\end{array}\right], \mathcal {A}=I+\epsilon A_{cl}^T.$

By some simple algebraic manipulations and using Schur complement
lemma, the  inequality (\ref{Eq:Thm 1 p4}) can also be rewritten as
\begin{equation}\begin{array}{l} \label{Eq:Thm 1 p5}\left[\begin{array}{cccc}- I+\gamma^{-2}D_{cl}^TD_{cl}
&\sqrt{\epsilon}B_{cl}^TQ\mathcal
{A}^T+\gamma^{-2}\sqrt{\epsilon}D_{cl}^TC_{cl}&\sqrt{\epsilon}B_{cl}^TQ\\{*}&\mathcal
{A}Q\mathcal {A}^T-Q+\gamma^{-2}\epsilon C_{cl}^TC_{cl}&0
\\{*}&{*}&-Q
\end{array}\right]<0,\end{array}
\end{equation}

which implies that the following inequalities hold.
\begin{equation*}\begin{array}{l}
\Sigma=\gamma^2 I-D_{cl}^TD_{cl}>0,\end{array}
\end{equation*}
\begin{equation*}\begin{array}{l}\left[\begin{array}{cc}W_1+\gamma^2W_2^T\Sigma^{-1}W_2&
0\\
* &
-Q\end{array}\right]+O(\epsilon)<0
\end{array}
\end{equation*}
where $W_1=A_{cl}^TQ+ QA_{cl}+\gamma^{-2}
C_{cl}^TC_{cl},W_2=B_{cl}^TQ+\gamma^{-2}D_{cl}^TC_{cl}^T$.

Obviously, when $\epsilon$ is sufficiently small, (\ref{Eq:Thm 1
p7}) holds.
\begin{equation}\label{Eq:Thm 1 p7}\begin{array}{l}\left[\begin{array}{cc}W_1+\gamma^2W_2^T\Sigma^{-1}W_2&
0\\
* &
-Q\end{array}\right]<0.
\end{array}
\end{equation}

Then,  by using the Schur complement lemma and some algebraic
manipulations,  the result in (\ref{Thm 1 p8}) is given.
\begin{equation}\label{Thm 1 p8}\left[\begin{array}{cccc} A_{cl}^TQ+ QA_{cl}&
 QB_{cl}&C_{cl}^T\\
 * &-I& D_{cl}^T\\ * & * &-\gamma^2 I\end{array}\right]<0.
\end{equation}

Let $J_1=\left[\begin{array}{cccc}
I&0&0\\0&0&I\\0&I&0\end{array}\right]$ and $P=Q^{-1}$. Then,
multiplying (\ref{Thm 1 p8}) on the right by $J_1^T$ and on the left
by $J_1$, (\ref{Thm 1 p9}) is obtained by some simple matrix
manipulations.
\begin{equation}\label{Thm 1 p9}\left[
\begin{array}{cccc}
PA_{cl}^T+A_{cl}P&PC_{cl}^T&B_{cl}\\
{*}&-\gamma^2I&D_{cl}\\
{*}&{*}&-I\\

\end{array}\right]<0.
\end{equation}

Further, based on duality theory, for the transfer function
$C_{cl}(s I-A_{cl})^{-1}B_{cl}+D_{cl}$ and its dual $B_{cl}^T(s
I-A_{cl}^T)^{-1}C_{cl}^T+D_{cl}^T$, the $H_{\infty}$ norms are
equivalent. Then, (\ref{Eq:Lem_3}) can be obtained.

This completes the proof.
 \end{proof}


\begin{rem}\label{re3}
By the introduction of additional matrix and a small sufficient
scalar, additional degrees of freedom can be obtained as in
\cite{JiaYM}, which may make the condition be of use in the solution
of many difficult control synthesis problems in a potentially less
conservative framework, and the linear part of the deadzone can be
effectively compensated.
\end{rem}



 \begin{Thm}\label{thm2}  For a prescribed scalar $\gamma>0$and a sufficiently small scalar $\epsilon>0$, the
 closed-loop sliding dynamics
 system in
(\ref{Eq:closed sliding mode dynamics}) is asymptotically stable,
and satisfies the norm constraint $\|T(s)\|_{\infty}<\gamma$ if and
only if
 there exist matrices $0<\tilde{Q}_s=Q_s^T=\left[\begin{array}{cc}\tilde{Q}_{sij}\end{array}\right]_{2\times2}\in
R^{2n\times 2n}$with $\tilde{Q}_{sij}$ having appropriate
dimensions,  and matrices $M$ and $G$ such that the following LMI
hold:

\begin{equation}\begin{array}{l}
\label{Eq:Thm 2 stand} \left[\begin{array}{ccccc}
\tilde{Q}_s-He\{G\}&G+\epsilon \Gamma_a &0&\sqrt{\epsilon}B_{cl}\\ *&-\tilde{Q}_s&\sqrt{\epsilon}\Gamma_c^T&0\\

*&*&-\gamma^2 I&0
\\
*&*&*& - I
\end{array}\right]<0,\end{array}
\end{equation}
where
\begin{eqnarray*}
&&\Gamma_a=A_{cl}G+Bm_{min}M, \Gamma_c=CG+DM,\\&&
\end{eqnarray*}
In addition, the constraints $\|T(s)\|_{\infty}<\gamma$ is satisfied
simultaneously for the closed-loop system with the stabilizing state
feedback gain matrix
\begin{equation}
\label{Eq:State Controller Para} K=MG^{-1}.
\end{equation}
\end{Thm}
 \begin{proof}
Following similar lines as in \cite{OliveiraMC02},  to make the
 problem tractable, first construct the matrix $Z^T$ as

\begin{equation}
\label{Eq:Thm 3
pr2}Z^T=\left[\begin{array}{cc}Y&N\\N&-N\end{array}\right].
\end{equation}
 Also introduce the
transformation matrix  $\mathcal {J}_1=(I_2\otimes E)\oplus I\oplus
I$ with $E=\left[\begin{array}{cc}I&I\\0&I\end{array}\right]$. Then,
pre- and post-multiply (\ref{Eq:thm 1 condition}) by $\mathcal
{J}_1$ and $\mathcal {J}_1^T$, respectively, and delete the second
and the forth   rows and columns, it follows that
\begin{equation}\begin{array}{l}
\label{Eq:Thm2 pr1} \left[\begin{array}{ccccc} (1,1)&
(1,2)&0&\sqrt{\epsilon}(Y+N)B_{cl}\\
 *&(2,2)&\sqrt{\epsilon}(C+DK)^T&0\\

*&*&-\gamma^2 I&0
\\ *&
*&*& - I
\end{array}\right]<0\end{array}
\end{equation}
where
\begin{eqnarray*}
&&(1,1)=Q_{s11}+He\{Q_{s12}\}+Q_{s22}-He\{Y+N\},\\&&
(1,2)=Y+N+\epsilon (Y+N)(A_{\delta}+Bm_{min}K),\\&&
(2,2)=-Q_{s11}+He\{Q_{s12}\}+Q_{s22}.
\end{eqnarray*}

Further, define $G^T=(Y+N)^{-1}$,
$\tilde{Q}_s=G^T(Q_{11}+He\{Q_{s12}\}+Q_{s22})G$ and $M=C_cG$, then,
pre- and post-multiply (\ref{Eq:Thm2 pr1}) by $(I_3\otimes
G^T)\oplus I\oplus I$ and $(I_3\otimes G)\oplus I\oplus I$,
respectively, (\ref{Eq:Thm2 pr1}) is equivalent to (\ref{Eq:Thm 2
stand}). This completes the proof.
 \end{proof}






\begin{Thm}\label{thm3}
For a prescribed sufficiently small constant $\mu>0$, the linear
continuous system (\ref{Eq:equivalent system}) subjected to
Assumptions 1-4 with multiple inputs containing actuator dead zone,
is asymptotically converge to the sliding mode (\ref{Eq:sliding
surface}), if the nonlinear control law is chosen as
(\ref{Eq:nonlinear Controller})
\begin{equation}
\label{Eq:nonlinear Controller}
\begin{array}{rcl}
u_1(x,t)&=&-\rho(t)(GB)^{-1}{s(x,t)\over \|s(x,t)\|}\\[1mm]
\end{array}
\end{equation}
where
\begin{eqnarray*}
\rho(x,t)={1\over
m_{min}}(\mu+\|GE\|\|{\bar{d}(x,t)}\|+\|GB\|\|{\bar{w}(x,t)}\|)
\end{eqnarray*}
\end{Thm}

 \begin{proof}
~Let us choose a $Lyapunov$ functional candidate
 \begin{equation*}
\label{EQ:lyapunov function} V(t) = {1\over 2} s^2
\end{equation*}
Calculating the difference of $V(t)$ along the system
(\ref{Eq:equivalent system}), we have
\begin{eqnarray*}
\dot
V(t)=s^T\dot{s}(x,t)&=&s^T(GB((m-{m_{min}})u_0+mu_l+d(t))+GEw(t))\\&&
\leq
\|s\|(-m_{min}\rho+\|GE\|\|{\bar{d}(x,t)}\|+\|GB\|\|{\bar{w}(x,t)}\|)\\&&<0
\end{eqnarray*}
 This completes the proof.
 \end{proof}


\begin{rem}\label{re5}
The chattering phenomenon of nonlinear controller design will be
serious. We can employ continuous boundary layer approximation of
discontinuous control presented in \cite{27,28} to eliminate it. But
the key point in this paper is not here, so we do not focus on it.
\end{rem}



Then, based on Theorem 2 and Theorem 3, the following algorithm is
presented for the design of integral sliding mode controller:

{\bf Algorithm 1.}
\begin{itemize}
\item[]{\bf Step 1.} Minimize $\gamma^2$ subject to the LMI
constraints (\ref{Eq:Thm 2 stand}). Denoting the optimal solution as
$M$,  $G$, and $\gamma$, then, the state-feedback controller gain
 can be obtained by $K=MG^{-1}$.
\item[]{\bf Step 2.} Substitute $K$ into (\ref{Eq:sliding
surface}). Then, the integral sliding surface can be given by
$s(x,t)=B^+[x(t)-x(t_0)-\int_{t_0}^{t_{f}}((A+BMG^{-1})x(\tau))\,dt]$.
\item[]{\bf Step 3.} Set the control as $u(x,t)=MG^{-1}x(t)-\rho{s(x,t)\over
 \|s(x,t)\|}$,where $\rho={1\over
m_{min}}(\mu+\|GE\|{\bar{d}}+\|GB\|{\bar{w}})$.
\end{itemize}


$\lim\limits_{T_f\rightarrow\infty}\int_{T_0}^{T_f}=1$\\
$\sum\limits_{i=1}^{2^m}$







\section{Numerical Simulations}
In this section,  a numerical example is proposed to show the
effectiveness of the proposed control strategy. For this purpose,
consider the continuous-time system  (\ref{Eq:equivalent system})
with
\begin{eqnarray}\label{Eq:Continuous-time System example}
&&A=\left[\begin{array}{cccc}0.05 &  -0.05  &  -0.23  &  -0.99\\
    1.67 & -0.78  & 0.77  & -1.53\\
   -1.46&    0.60  & -2.97  & -1.06\\
   -1.78  &  1.49  &  1.69  &  0\end{array}
\right], B=\left[\begin{array}{cc}1.27 &   0.18\\
    0.63  &  2\\
   -1.83  &  1.39\\
    0.52 &   1.92\end{array}
\right],
E=\left[\begin{array}{ll}-1.3 &  -0.1\\
   -1.87 &   0\\
    1.79  & -0.4\\
   -0.95 &   1.76\end{array}
\right],\nonumber\\[1mm]&&
C=\left[\begin{array}{cccc}-1.97 &   1.04 &   1.26& 0.16\end{array}
\right], D_{21}=\left[\begin{array}{cc}0.4  &
1.28\end{array}\right].
\end{eqnarray}



It need to point out that we always set $\epsilon=0.02$, and $m=1.2$
in this example.


In the simulation, the bounds of the parameters of the dead-zone are
$m_{\min }=0.85$, $m_{\max}=1.25$, $b_{r\min}=0.1, b_{r\max}=0.6,
b_{l\min}=-0.7, b_{l\max}=-0.1 $, then we can calculate the $\bar
d(u(t))=0.875$.



Minimizing $\gamma$ subject to (\ref{Eq:Thm 2 stand}) in Theorem
\ref{thm2}, the $H_{\infty}$ state feedback controller's gain matrix
$K$ is obtained as follows by using (\ref{Eq:State Controller
Para}),

\begin{equation*}
\begin{array}{c} K=\left[\begin{array}{cccc}
   -6.6234& -1.8059&  3.0005& 0.7187\\
   3.6089 &  -0.2482 &  -1.9220&   -0.3496 \end{array}\right]
\end{array}
\end{equation*}
and the performance indexes $\gamma$ is obtained as $4.2995e-007$.
Then, from Theorem \ref{thm3}, we get nonlinear control law
parameter $\rho=1.1301$.


To illustrate the simulation results of the control objectives, with
the zero initial state, the disturbance
$w^T(t)=\left[\begin{array}{cc}w_1^T(t)&w_2^T(t)\end{array}\right]$
added to the system is assumed to be:
\begin{equation}
\label{EQ:Disturbance input}
\begin{array}{l}w_1(t)=w_2(t)=\left\{\begin{array}{llc} 0.2\, cos(t)
\,\,\,\, 20\leq t\leq 30\\0\,\,\,\,
otherwise.\end{array}\right.\end{array}
\end{equation}

From Fig.1, the simulation results  show that the system is more
robust than using $H_{\infty}$ technology alone when the unmatched
disturbance is added to the system. It can also be concluded that
the system states can be affected severely when the deadzone fault
occurs using traditional integral sliding mode controller, while the
system is on the sliding mode all the time if the deadzone failure
occurs using the proposed scheme.


From Fig.2, the closed loop system is on the integral sliding
surface whether the deadzone or unmatched disturbances occurs using
proposed scheme. While the system will be out of the sliding surface
using traditional scheme.
%%
%%\begin{center}
%%\includegraphics [scale=0.5,trim=0.5 0.5 0.5 0.5]{controlvariable.eps}\\
%%\label{Fig.4}Fig.~3 ~~ Control variables
%%\end{center}

\section{Conclusion}

In this paper, robust integral sliding mode control has been
proposed for multi inputs systems with nonsymmetric deadzone and
unmatched disturbances. Considering unmatched perturbations, the
state feedback controller in combination with ISM surface has been
proposed to further robustify than using $H_\infty$ alone. As far as
the case of deadzone is concerned,  the integral sliding mode
controller ensures that, even in the presence of uncertainties, the
deadzone can be compensated completely. Finally, simulation studies
are presented to illustrate the effectiveness of the proposed
control.





















\section*{Acknowledgement}




\begin{thebibliography}{99}



\bibitem{1}
G. Tao and P. Kokotovic, Adaptive Control of Systems With Actuator
and Sensor Nonlinearities. New York: Wiley, 1996.


\bibitem{2}

Recker,D., Kokotovic, P.V., Rhode,D., Winkelman,J. Adaptive
nonlinear control of systems containing a dead-zone.{\sl Proceedings
of the 30th IEEE conference on decision and control}.IEEE: 1991.
2111-2115.

\bibitem{3}
Tao,G.,Kokotovic,P.V. A daptive control of plants with unknown
dead-zones.{\sl IEEE Transactions on Automatic Control},1994
{\bf39}: 59-68.

\bibitem{4}
Wang,X.S., Su,C.Y., Hong,H. Robust adaptive control of a class of
linear systems with unknown dead-zone.{\sl Automatica},2004, {\bf
40}: 407-413.

\bibitem{5}
H. J. Ma, G. H. Yang, Adaptive output control of uncertain nonlinear
systems with non-symmetric dead-zone input.{\sl Automatica},
2010,{\bf 46}:  413-420.

\bibitem{6}
Zhou,J., Wen,C., Zhang,Y. Adaptive output control of nonlinear
systems with uncertain dead-zone nonlinearity.{\sl IEEE Transactions
on Automatic Control}, 2006,{\bf 51}: 504-511.

\bibitem{7}
Zhou,J. Decentralized adaptive control for large-scale time-delay
systems with dead-zone input, {\sl Automatica},2008,{\bf 44}:
1790-1799.

\bibitem{8}
Ibrirs., Xie W.F., Su C.Y,  Adaptive tracking of nonlinear systems
with non-symmetric dead-zone input,{\sl Automatica},2007, {\bf
43}(3):522-530.

\bibitem{9}
Zhang T.P., Ge S.S, Adaptive dynamic surface control of nonlinear
systems with unknown deadzone in pure feedback form, {\sl
Automatica}, 2008,{\bf 44}: 1895-1903.

\bibitem{21}
Selmic R.R.,Lewis F.L. Dead-zone compensation in motion control
systems using neural networks. {\sl IEEE Transactions on Automatic
Control},2000, {\bf 45}:602-613.

\bibitem{10}
Mou Chen, Shuzhi Sam Ge, Bernard Voon Ee How, Robust Adaptive Neural
Network Control for a Class of Uncertain MIMO Nonlinear Systems With
Input Nonlinearities, {\sl IEEE Transactions on neural networks},
2010, {\bf 21}(5):796 - 812 .

\bibitem{11}
Jang,J.O. A dead-zone compensator for a DC motor system using fuzzy
logic control.{\sl IEEE Transactions on Systems,Man and Cybernetics
Part C: Applications and Reviews},2001, {\bf 31}, 42-47.

\bibitem{12}
Kim,J.H., Park,J.H., Lee,S.W., Chong,E.K.P. A two-layered fuzzy
logic controller for systems with dead-zones. {\sl IEEE Transactions
on Industrial Electronics}, 1994,{\bf 41}, 155-162.

\bibitem{13}
F.L.Lewis, K.Liu, R.Selmic,  L.-X.Wang, Adaptive fuzzy logic
compensation of actuator deadzones, {\sl J.Robot.Syst.},1997, {\bf
14}:501-511.



\bibitem{15}
M.L.Corradini, G.Orlando. Robust stabilization of nonlinear
uncertain plants with backlash or deadzone in the actuator,{\sl IEEE
Trans. Contr. Syst. Technol.}, 2002,{\bf 10}: 158-166 .

\bibitem{16}
M.L.Corradini, G.Orlando, G.Parlangel. A VSC Approach for the Robust
Stabilization of Nonlinear Plants With Uncertain Nonsmooth Actuator
Nonlinearities¡ªA Unified Framework, {\sl IEEE Transaction on
automatic control}, 2004, {\bf 49}(5):807-812.



\bibitem{shyu05}
Kuo-Kai Shyu ,Wen-Jeng Liu ,Kou-Cheng Hsu. Design of large-scale
time-delayed systems with dead-zone input via variable structure
control,{\sl Automatica}, 2005 , {\bf41}:1239¨C1246.

\bibitem{18}
Wen-Jeng Liu, Kuo-Kai Shyu,  Kou-Cheng Hsu. Design of Uncertain
Multi-input Systems with State Delay and Input Deadzone Nonlinearity
via Sliding Mode Control, {\sl International Journal of Control,
Automation, and Systems}, 2011,{\bf 9}(3):461-469.

\bibitem{19}
Kou-Cheng Hsu, Wei-Yen Wang,Ping-Zong Lin, Sliding Mode Control for
Uncertain Nonlinear Systems With Multiple Inputs Containing Sector
Nonlinearities and Deadzones, {\sl IEEE Transaction on
systems,man,and sybernetics-part B:sybernetics},2004, {\bf 34},
(1):374-380.

\bibitem{20}
V. I. Utkin, Sliding Modes in Control Optimization. New York:
Springer-Verlag, 1992.

\bibitem{xluo09}
X.Luo, X.Wu, X.Guan, Adaptive backstepping fault-tolerant control
for unmatched non-linear systems against actuator dead-zone, {\sl
IET Control Theory and Applications}, 2010, {\bf 4|}(5):879-888.



\bibitem{22}

S.Aloui, O.Pages, A.ElHajjaji, A.Chaari, Y.Koubaa, Improved fuzzy
sliding mode control for a class of MIMO nonlinear uncertain and
perturbed systems,{\sl J. Appl.Soft Comput.}, 2011, {\bf 11}(1):
820-826.

\bibitem{23}
V.I.Utkin and J.Shi, Integral sliding mode in system so perating
under uncertainty conditions,{\sl presented at the 35th
Conf.Decis.Control},1996.



\bibitem{24}
F.Castanos, L.Fridman, Analysis and design of integral sliding
manifolds for systems with unmatched perturbations,{\sl IEEE
Trans.Autom. Control},2006, {\bf 51}(5):853-858.

\bibitem{25}
W.J.Cao, J.X.Xu, Nonlinear integral-type sliding surface for both
matched and unmatched uncertain systems, {\sl IEEE
Trans.Autom.Control}, 2004, {\bf 49}(8):1355-1360.

\bibitem{26}
Jeang-Lin Chang, Dynamic Output Integral Sliding-Mode Control With
Disturbance Attenuation, {\sl IEEE Trans.Autom.Control},2009, {\bf
54}(11):2653-2658.

\bibitem{27}
Young, K.D., Utkin, V.I., Ozguner, U., A control engineer's guide to
sliding mode control,  {\sl IEEE Transactions on Control Systems
Technology }, 1999, {\bf 7}(3):328-342 .

\bibitem{28}
C.Edwards and S.K.Spurgeon, Sliding Mode Control Theory and
Application. London, U.K.: Taylor  Francis, 1998.


\bibitem{GahinetP}
Gahinet P,   Apkarian P.  A linear matrix inequality  approach to
$H{\infty}$ Control. {\sl International  Journal of  Robust
Nonlinear Control}, 1994, {\bf 4}(5): 421-448




\bibitem{JiaYM}
 Jia Y M. Alternative  proofs for improved  LMI  representations
for the analysis and  the  design  of continuous-time systems with
polytopic uncetainty: A predictive approach. {\sl IEEE Transactions
on Automatic Control}, 2003, {\bf 48}(8): 1413-1416.


\bibitem{OliveiraMC02}
de Oliveira M C,Bemussou J,Geromel J C, Extended $H_2$ and $H_{\infty}$
norm characterizations and controller parametrizations for
discrete-time systems. {\sl International Joural of Control}, 2002,
{\bf 75}(9):666-679.



\bibitem{Gahinet96}
Gahinet P.  Explicit controller formulas for LMI-based $H_{\infty}$
synthesis. {\sl Automatica}, 1996, {\bf 32}(7):  1007-1014




\end{thebibliography}



\end{document}
